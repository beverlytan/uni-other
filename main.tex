\documentclass[a4paper, 12pt]{article}
\usepackage[utf8]{inputenc}
\usepackage{color}
\usepackage{parskip}
\usepackage{hyperref}
\usepackage{graphicx}

\begin{document}

\title{Testing Latex}
\author{Beverly Tan}
\date{27 February 2019}
\maketitle

\pagenumbering{roman}
\tableofcontents
\newpage
\pagenumbering{arabic}

\section{Introduction}

This is my first document using Overleaf \LaTeX

\section{Basic material}

\subsection{The very basics}
\label{stage1}

This is the methodology for stage 1. 

How do I make the paragraphs look like what we see now, because the original default paragraph looks horrible? I use the "usepackage\{parskip\}" at the start before "begin\{document\}". 

\subsection{Sectioning and referencing}

Refer to section \ref{stage1} on page \pageref{stage1} for methodology for stage 1. In this case the "2.1" and "(page) 1" is not fixed, we've specified it by "ref\{stage1\}" and "pageref\{stage1\}" respectively, which we have stated above under the section we wanted to be referenced with "label\{stage1\}" - in this case that line of code is actually under "The very basics". So if we were to add a section between the Introduction and Basic material, it would automatically change it to 3.1 and you wont have to do that manually! :-) 

Another cool thing about \LaTeX is that using the "hyperref" package at the start of the document - our contents page, and these references now become clickable! :-) 

\newpage

\section{Formatting}

\subsection{Font effects}

This is me testing out different font effects. \textit{This text will be in italics.} \textbf{This text will be bolded.} \underline{This text will be underlined.} {\color{blue}{This text will be blue in colour.}} {\color{red}{This text will be red in colour.}} {\tiny These are tiny words.}{\large These are larger words.} What if I want to combine the different text effects? {\color{red}{\textbf{\textit{This text will be in bolded italics in red font.}}}}

\subsection{Lists and spaces}

List number 1

\begin{enumerate}
    \item Hello
    \item How are you?
    \begin{itemize}
        \item This makes smaller things
        \item How cool! 
    \end{itemize}
\end{enumerate}

\vspace{12pt}

List number 2

\begin{itemize}
    \item Default bullet point
    \item[-] Line bullet point
    \item[(a)] Word bullet point
\end{itemize}

\vspace{12pt}

"vspace\{12pt\}" will create a space that is of 12pt size above this line.

\subsection{Special characters}

Sometimes you will need to add "\textbackslash" before some symbols such as \# or \&, because on their own they mean certain things in \LaTeX, so those would look like this: \textbackslash\# and \textbackslash\&. 

Item \#1A\textbackslash642 costs \$8 \& is sold at a \~{}10\% profit.

\newpage

\section{Tables}

In this section, I will now try to create tables. Note that the default for tables is that they have no lines, if you want lines you will have to specify that. 

\begin{table}[h]
    \centering
    \begin{tabular}{|c|c|}
        \hline
        8 & here's \\
        \cline{2-2}
        86 & stuff \\
        \hline
         \hline
         88 & now \\
         \hline
    \end{tabular}
    \caption{Test table number one}
\end{table}

Note that if I want my table to be at a specific location relative to the text, instead of \LaTeX deciding where the tables will be placed, next to the \{table\}, I will have to add [h] - h for "here", to tell \LaTeX that I want it exactly at this spot. This is a problem when you wrap your whole table in "table", instead of just using "tabular" - because wrapping it in table makes it an environment and allows it to float! 

\vspace{12pt}

\begin{table}[h]
    \centering
    \begin{tabular}{l|r|r}
        Item & Quantity & Price(\$) \\
        \hline
        Nails & 500 & 0.34 \\
        Wooden boards & 100 & 4.00 \\
        Bricks & 240 & 11.50
    \end{tabular}
    \caption{Price list of construction items}
\end{table}

Let me try out another sample table! :-) 

\begin{table}[h]
    \centering
    \begin{tabular}{l|ccc}
        & & Year \\
        \cline{2-4}
        City & 2006 & 2007 & 2008 \\
        \hline
        London & 123 & 456 & 789 \\
        Berlin & 098 & 765 & 432
    \end{tabular}
    \caption{Caption}
    \label{tab:my_label}
\end{table}

Great, it worked!

\newpage

\section{Figures}

Now I want to try to insert figures into my document. The first thing to do is to upload an image into this latex-intro folder that I have created, which will be uploaded into the cloud. 

\begin{figure}[h!]
    \centering
    \includegraphics[width=1\textwidth]{DSC01138.JPG}
    \caption{Soda}
    \label{fig:my_label}
\end{figure}

Heh soda is v cute!

\newpage

\section{Equations}

Let us check out what math mode does: $1 + 2 = 3$. Here I use a single \$ at the start and end of the "1+2=3". This is as compared to the following, where I use double \$\$ at the start and front of the "1+2=3", which makes it into a separate line, and centred: $$1 + 2 = 3$$

If I wanted to make a numbered eqn, I have to use the following: \\
\textbackslash begin\{equation\} ... \textbackslash end\{equation\}, which will end up looking like this: 

\begin{equation}
    1 + 2 = 3
\end{equation}

If I want to make a series of questions, you use "eqnarray" instead of "equation" (or you could always just stack two equations, but that's less organized). 

\begin{eqnarray}
    1 + 2 = 3 \\
    4 + 5 = 9
\end{eqnarray}

\subsection{Mathematical symbols}

Note that when we insert mathematical symbols, we also have to be in math mode - which means that that bit of code/text should be within \$ or \$\$!

\begin{equation}
e = mc^2 
\end{equation}

\begin{equation}
\pi = \frac{c}{d}
\end{equation}

\begin{equation}
\frac{d}{dx} e^x = e^x
\end{equation}

\begin{equation}
\frac{d}{dx} \int_0^\infty f(s)ds = f(x)
\end{equation}

\begin{equation}
f(x) = \sum_{i} = 0^\infty \frac{f^(^i^)(0)}{i!} x^i
\end{equation}

\begin{equation}
x = \sqrt{\frac{x_i}{z} y}
\end{equation}

\end{document}


